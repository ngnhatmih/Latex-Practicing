\documentclass{article}
\usepackage{amsmath}
\usepackage{array}
\usepackage[top=1.5cm, bottom= 2cm]{geometry}
\usepackage[utf8]{vietnam}
\usepackage{fancyhdr}
\usepackage{graphicx}
\usepackage{systeme}
\usepackage[overload]{empheq}
\usepackage{amssymb}
\usepackage{xcolor}

%%%%%%HEADER-AND-FOOTER%%%%%%%
\fancyhead[OR, R]{\today}
\fancyhead[OC, C]{\textbf{Linear Algebra - 2}}
\fancyhead[OL, L]{Nguyễn Nhật Minh - 22010265}
\fancyfoot[OL, L]{}
\fancyfoot[OC, C]{\thepage}
\fancyfoot[OR, R]{}
%%%%%%%%%%%%%%%%%%%%%%%%%%%%%%

\begin{document}
\pagestyle{fancy}
\renewcommand{\headrulewidth}{0.4pt}
\renewcommand{\footrulewidth}{0.4pt}

\large
\renewcommand{\arraystretch}{1.2}


\section[Câu 1]{{\large\normalfont(3.5 điểm). Cho ma trận sau}}
\begin{flushleft}
	\begin{equation*}
		A = \begin{bmatrix}
		-3 & 2 & m \\
		6 & 2 & -1 \\
		4 & 0 & -3
		\end{bmatrix}
	\end{equation*}
	    
	\subsection{Tìm điều kiện của m để ma trận trên khả nghịch}
	Để ma trận A khả nghịch thì định thức A phải khác 0 tương đương với \colorbox{gray!20}{$\det A \neq 0$}
	\begin{center}
		\begin{table}[h]
			\centering
			\begin{tabular}{c|c}
				    	
				Matrix & Row operations
				\\ 
				\hline    	 
				\\
				$\begin{bmatrix}
				-3 & 2           & m               \\
				6  & 2           & -1              \\
				4  & 0           & -3              
				\end{bmatrix}$  \\ \\
				$\begin{bmatrix}
				-3 & 2           & m               \\
				0  & 6           & 2m-1            \\
				0  & \frac{8}{3} & \frac{4m-9}{3}  
				\end{bmatrix}$ & $R_2+R_1 \to R_2$, $R_3+\frac{4}{3}R_1 \to R_3$ \\ \\
				$\begin{bmatrix}
				-3 & 2           & m               \\
				0  & 6           & 2m-1            \\
				0  & 0           & \frac{4m-23}{9} 
				\end{bmatrix}$ & $R_3 - \frac{4}{9}R_2 \to R_3$
			\end{tabular}
		\end{table}
		\begin{align*}        
			\det A &= \begin{vmatrix}
			-3 & 2 & m               \\
			6  & 2 & -1              \\
			4  & 0 & -3              
			\end{vmatrix}
			= \begin{vmatrix}
			-3 & 2 & m               \\
			0  & 6 & 2m-1            \\
			0  & 0 & \frac{4m-23}{9} 
			\end{vmatrix} 
			= -8m + 46 \neq 0 
		\end{align*} \\[2.5pt]
		\colorbox{gray!20}{$\longrightarrow m \neq 5.75$}
	\end{center}
	
\end{flushleft}
\subsection{Nếu m là chữ số cuối cùng của mã sinh viên của bạn thì ma trận trên có khả nghịch không? Nếu có hãy tìm định thức của ma trận $A^{-1}$}
\begin{flushleft}
	Với m là chữ số cuối cùng là 5 thay vào định thức A ta được:  \\
	\begin{equation}
		\det A = -8 \times 5 + 46 = 6 \neq 0
		\label{matrix A}
	\end{equation} 
	   
	Đồng nghĩa với ma trận A là khả nghịch. Từ \eqref{matrix A} ta có:
	\begin{alignat*}{2}
		  & A^{-1} &   & = &   & \begin{bmatrix} 
		C_{11} & C_{12} & C_{13} \\
		C_{21} & C_{22} & C_{23} \\
		C_{31} & C_{32} & C_{33}
		\end{bmatrix} \frac{1}{\det A} \\
		  &        &   & = &   & \begin{bmatrix} 
		-6 & 6 & -12 \\
		14 & -11 & 27 \\
		-8 & 8 & -18
		\end{bmatrix} \frac{1}{6}
	\end{alignat*}
\end{flushleft}
\pagebreak

\section[Câu 2]{{\large\normalfont(3.5 điểm). Giải hệ phương trình sau bằng phương pháp khử gauss:}}
\begin{alignat*}{4}[left = \empheqlbrace]
	x   & {}+{} & 4y & {}+{}  & 2z    & {}-{} & 5t  & {}={} & 5 \\
	-2x & {}+{} & 5y & {} +{} & 3z {} & -     & {}t & {}={} & 5 \\
	3x  & {}-{} & y  & {}-{}  & z     & {}-{} & 4t  & {}={} & 7 
\end{alignat*}

Từ hệ phương trình trên ta lấy ma trận mở rộng à \\
\begin{equation*}
	\text{Ã} =  \begin{bmatrix}
	1 & 4 & 2  & -5 & \Bigm| & 5 \\
	-2 & 5 & 3 & -1 & \Bigm| & 5 \\
	3 & -1 & -1 & -4 & \Bigm| & 7
	\end{bmatrix}
\end{equation*}

\begin{center}
	\begin{table}[h]
		\centering
		\begin{tabular}{c|c}
			    	
			Augmented matrix & Row operations
			\\ 
			\hline    	 
			\\
			$\begin{bmatrix}
			1 & 4   & 2  & -5  & \Bigm| & 5  \\
			0 & 13  & 7  & -11 & \Bigm| & 15 \\
			0 & -13 & -7 & 11  & \Bigm| & 8  
			\end{bmatrix}$ & $R_2+2R_1 \to R_2$, $R_3 -3R_1 \to R_3$ \\ \\
			$\begin{bmatrix}
			1 & 4   & 2  & -5  & \Bigm| & 5  \\
			0 & 13  & 7  & -11 & \Bigm| & 15 \\
			0 & 0   & 0  & 0   & \Bigm| & 7  
			\end{bmatrix}$ & $R_3+ R_2 \to R_3$
		\end{tabular}
	\end{table}
\end{center}

\begin{alignat*}{4}[left = \empheqlbrace]
	x & {}+{} & 4y  & {}+{}  & 2z    & {}-{} & 5t   & {}={} & 5  \\
	  & {}{}  & 13y & {} +{} & 7z {} & -     & {}11 & {}={} & 15 \\
	  & {}{}  &     & {}{}   &       & {}{}  & 0    & {}={} & 7  
\end{alignat*}
Từ đây dễ thấy hệ phương trình \textbf{vô nghiệm}

\section[Câu 3]{\large\normalfont (3 điểm). Trong không gian vector $\mathbb{R}^4$ hãy kiểm tra xem vector $\mathbf{u} = (6, -2, 0, 5)$ có phải là tổ hợp tuyến tính của ba vector sau đây không?}
\begin{align*}
	\mathbf{b}_1 & = (3, 9, -4, -2) \\
	\mathbf{b}_2 & = (2, 3, 0, -1)  \\
	\mathbf{b}_3 & = (2, -1, 2, 1)  \\
\end{align*}
\begin{flushleft}
	Để \mathbf{u} là tổ hợp tuyến tính của 3 vector trên thì ta xét \colorbox{gray!20}{$\mathbf{u} = a\mathbf{b}_1 + b\mathbf{b}_2 + c\mathbf{b}_3$}. Lúc này đề bài trở thành giải hệ phương trình ẩn a, b, c: 
	\begin{alignat*}{3}[left = \empheqlbrace]
		3a  & {}+{} & 2b & {}+{}  & 2c & {}={} & 6  \\
		9a  & {}+{} & 3b & {} -{} & c  & {}={} & -2 \\
		-4a & {}{}  &    & {}+{}  & 2c & {}={} & 0  \\
		-2a & {}-{} & b  & {}+{}  & c  & {}={} & 5  
	\end{alignat*}
	Tương tự cách làm như bài 3, kết quả cuối cùng suy ra được hệ phương trình vô nghiệm đồng nghĩa $\mathbf{u}$ không phải là tổ hợp tuyến tính của 3 vecto.
\end{flushleft}


\end{document}
